\documentclass[a4paper,11pt]{report}
\usepackage[T1]{fontenc}
\usepackage[utf8]{inputenc}
\usepackage{lmodern}
\usepackage[francais]{babel}

\title{Projet PAD}
\author{Olivier Boissard, Kevin Boulala, Maxime Dubois, Antoine Lavier}
\date{Dernière modification : \today}

\begin{document}

\maketitle
\tableofcontents

\chapter{Contexte}
\section{Description générale du projet}
Ce serait un jeu multijoueur par navigateur.\\
Dans ce jeu le joueur contrôlerait et managerait un vaisseau, le but étant de le faire grossir et prospérer.
L’équipage du vaisseau est composé de petits monstres capables de piloter le vaisseau en lui même, que de gérer les moteurs, utiliser les armes embarquées, etc. Pour faire prospérer son vaisseau, il faut utiliser son équipage comme une ressource de construction. Par exemple, le joueur souhaite avoir une arme plus puissante, le coût pour améliorer l’arme c’est de fusionner 2 membres de l’équipage. De même pour l’agrandissement du vaisseau.\\
Pour améliorer son vaisseau, il faut explorer des planètes pour «recruter» d’autres membres.Il faudra trouver le bon équilibre pour faire grossir son vaisseau et le rendre ainsi plus puissant, mais aussi d’avoir un équipage suffisamment important pour que chaque fonctionnalité du vaisseau puisse être exploité à son plein potentiel.\\
Les joueurs pourraient se rencontrer et organiser des batailles purement statistiques. Le vainqueur repartirait avec potentiellement quelques dégâts mineurs, mais surtout avec de nouveaux membres d’équipages. Le perdant, lui repartira avec un bout de son vaisseau amputé.
\section{Positionnement par rapport aux autres jeux par navigateur}
\chapter{Définitions des éléments constituant du jeu}
\section{L'univers}
\section{Le vaisseau}
\chapter{Expression fonctionnelle}
\section{Graphe des intéractions}
\section{Les fonctions principales}
\section{Les fonctions secondaires}
\end{document}
