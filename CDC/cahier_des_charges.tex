\documentclass[a4paper,11pt]{report}
\usepackage[T1]{fontenc}
\usepackage[utf8]{inputenc}
\usepackage{lmodern}
\usepackage[francais]{babel}

\title{Projet PAD}
\author{Olivier \bsc{Boissard}, Kevin \bsc{Boulala},\\
Maxime \bsc{Dubois}, Antoine \bsc{Lavier}}
\date{Dernières modifications : \today}

\begin{document}

\maketitle
\tableofcontents

\chapter{Contexte}
    \section{Description générale du projet}
    \paragraph{}
    C'est un jeu multijoueur par navigateur.
    \paragraph{}
    Dans ce jeu le joueur contrôle et gère un vaisseau, le but étant de le faire grossir et prospérer.
    \paragraph{}
    L’équipage du vaisseau est composé de petits monstres capables aussi bien de piloter le vaisseau en lui même, que de gérer les moteurs, utiliser les armes embarquées, etc. Pour faire prospérer son vaisseau, il faut utiliser son équipage comme une ressource de construction. Par exemple, le joueur souhaite avoir une arme plus puissante, le coût pour améliorer l’arme c’est de fusionner 2 membres de l’équipage. De même pour l’agrandissement du vaisseau.
    \paragraph{}
    Pour améliorer son vaisseau, il faut explorer des planètes pour «recruter» d’autres membres. Il faudra trouver le bon équilibre pour faire grossir son vaisseau et le rendre ainsi plus puissant, mais aussi d’avoir un équipage suffisamment important pour que chaque fonctionnalité du vaisseau puisse être exploité à son plein potentiel.
    \paragraph{}
    Les joueurs peuvent se rencontrer et organiser des batailles purement statistiques. Le vainqueur repartirait avec potentiellement quelques dégâts mineurs, mais surtout avec de nouveaux membres d’équipages. Le perdant, lui repartira avec un bout de son vaisseau amputé.

    \section{Positionnement par rapport aux autres jeux par navigateur}
    \begin{itemize}
      \item On vide la concurrence directe de Ogame.
      \item Le jeu vise une clientèle large, des joueurs occasionnels aux joueurs hardcores :
      \begin{itemize}
        \item Les mécanismes du jeu sont simples et tout gravite autour d'une seule ressource, ainsi il est simple d'accès pour tous. Ceci permettant au joueur de faire de courtes ou longues cessions de jeu.
        \item Il n'est pas seulement simple, le fait d'avoir une micro-gestion de son équipage amènera les joueurs les plus hardcores à vouloir créer le vaisseau le plus puissant.
      \end{itemize}
    \end{itemize}

\chapter{Définitions des éléments constituant du jeu}
    \section{L'univers}
      \begin{description}
        \item[Etoiles] ce sont des objets extrêmements chauds et hostiles aux monsters et à leur vaisseau.
        \item[Planètes habitées] ce sont des planètes où des monsters vivent.
        \item[Planètes hostiles] ce sont des planètes très hostiles, où les monsters n'ont pu s'y installer pour diverses raisons.
      \end{description}
      
    \section{Le vaisseau}
      Voici la description des différents modules qui peuvent être présent dans un vaisseau :
      \begin{description}
        \item[Réacteur] c'est la pièce centrale permettant à tout le vaisseau d'avoir la sauce nécessaire ! Et c'est une pièce puisant son énergie des précédents monsters fusionnés.
        \item[Poste de pilotage] c'est ici que l'on dirige le vaisseau.
        \item[Propulseur] c'est le module permettant de faire avancer le vaisseau.
        \item[Bouclier] il permet d'avoir une protection globale qui se régénère avec le temps.
        \item[Radar] ce module permet de détecter plus ou moins efficacement les éléments entourant le vaisseau (d'autres vaisseaux, des planètes, etc).
        \item[Arme] il existe différent types d'armes selon l'objectif : certaines sont efficaces contre les boucliers, d'autres plus pour détruire la coque.
      \end{description}
      Pour l'amélioration ou l'ajout de module, il sera nécessaire de fusionner des monsters. Par exemple, nous avons un vaisseau avec un équipage de 10 monsters. Ils sont tous occupés à une tâche sauf 2. On en utilisera un premier pour le fusionner avec le vaisseau et rajouter un propulseur, et le second sera affecté à l'entretien de ce nouveau module. Ainsi le vaisseau gagnera en vitesse.

\chapter{Expression fonctionnelle}
    \section{Graphe des intéractions}

    \section{Les fonctions principales}

    \section{Les fonctions secondaires}

\end{document}
