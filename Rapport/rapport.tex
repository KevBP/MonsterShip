\documentclass[a4paper,11pt]{report}
\usepackage[T1]{fontenc}
\usepackage[utf8]{inputenc}
\usepackage{lmodern}
\usepackage[francais]{babel}
\usepackage{graphicx}
\usepackage{setspace}
\usepackage{float}

\title{MonsterShip}
\author{Olivier \bsc{Boissard}, Kevin \bsc{Boulala}, Maxime \bsc{Dubois}, Antoine \bsc{Lavier}}

\begin{document}

\maketitle
\setcounter{tocdepth}{1}
\tableofcontents

\chapter{Introduction}
  Ce rapport parle du nouveau jeu de stratégie spatiale \textit{MonsterShip}. 
  Ce jeu permet de contrôler un vaisseau spatial, et de le déplacer dans l'espace afin d'attaquer d'autres joueurs, ou d'explorer des planètes.
  Le joueur pourra développer son vaisseau pour le rendre plus puissant, ou endommager les vaisseaux des autres pour diminuer leurs capacités.
  Une bonne combinaison des deux permettra d'avoir le vaisseau le plus puissant de la galaxie. 
  L'amélioration du vaisseau nécessitant d'avoir un équipage de plus en plus grand, il faudra aussi explorer des planètes pour recruter du personnel.
  Pour éviter qu'un joueur domine tous les autres, il disposera d'un certain nombre de points d'actions. 
  Ces points seront nécessaires pour déplacer le vaisseau ou recruter des membres d'équipage. 
  Dans le cas des membres d'équipage, une autre solution sera de rester plus longtemps sur la planète.

\chapter{Contexte}
  Ce jeu sera accessible par navigateur. Il sera déployé sur un serveur dédié, permettant d'avoir taux de disponibilité le plus élevé possible.
  Pour pouvoir participer, les joueurs disposeront d'un abonnements d'un ou plusieurs mois, et pourront s'ils le désirent acheter des bonus.
  

\chapter{Description du Jeu}
  Ce jeu de stratégie se déroule dans l'espace. Le joueur contrôle un vaisseau spatial alien, avec un équipage de monstres. 
  
  
  \section{Jeu de stratégie spatiale}
  \section{Jeu par navigateur}
  \section{Accessible à tous}
  \section{Elements du jeu}
\chapter{Fonctionnement du jeu et conseils}
  \section{Points d'actions}
    \subsection{Gain tout les jours}
    \subsection{Achat de points d'actions}
  \section{Déplacements du vaisseau}
  \section{Améliorations du vaisseau}
  L'équipage du vaisseau peut fusionner avec le vaisseau, afin de l'agrandir, ou avec les différents composants du vaisseau, pour augmenter ses capacités.
  Ceci doit permettre au joueur de gagner les combats face aux autres joueurs.
    \subsection{Agrandissement du vaisseau}
      Pour pouvoir améliorer les modules du vaisseau, ou en ajouter de nouveaux, il faut avoir suffisemment de place. 
      Pour cela, le joueur peut décider de fusionner une partie de son équipage avec son vaisseau. 
      Cette action lui permettra de gagner de la place dans le vaisseau mais en sacrifiant une partie de l'équipage.
      
    \subsection{Ajout de modules}
      Une fois que le vaisseau dispose de suffisemment de place, le joueur peut décider d'ajouter des modules (réacteurs, armes...). 
      Ces modules lui permettront d'améliorer son vaisseau afin de le rendre plus puissant.
          
    \subsection{Amélioration des modules}
      Après avoir ajouté des modules, il est possible de les améliorer, en les fusionnant avec des membres d'équipages. 
      Le joueur doit à nouveau sacrifier des monstres pour cela, mais il pourra alors vaincre d'autres joueurs pour en récupérer.
      
    \subsection{Affectation de l'équipage sur les composants du vaisseau}
      Enfin, les modules ont besoin de membres d'équipage pour fonctionner. 
      Ainsi, il faut s'organiser pour améliorer la taille du vaisseau, ajouter des modules et les améliorer, tout en gardant suffisemment de monstres pour pouvoir utiliser toutes les capacités du vaisseau.
          
  \section{Combats spatiaux}
  \section{Kidnapping de monstres}
    \subsection{Sur une planète}
      \paragraph{En fonction des points d'actions}
      \paragraph{En fonction du temps resté sur la planète}
    \subsection{Durant un combat}
  \section{Abandon du vaisseau}
\chapter{Prix}
  \section{Abonnements}
  \section{Bonus}
\chapter{Conclusion}

\end{document}
